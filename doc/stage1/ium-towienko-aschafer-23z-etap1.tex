% !TeX spellcheck = pl_PL
\documentclass[10pt,a4paper]{article}

\usepackage{geometry}
\geometry{
	a4paper,
	total={170mm,257mm},
	left=20mm,
	top=20mm,
}


% większość można wywalić, przekopiowane z projektu z MNUMów
\usepackage{mathtools}
\usepackage{polski}
\usepackage{amsfonts}
\usepackage{amsmath}
\usepackage[T1]{fontenc}
\usepackage{listings}
\usepackage{booktabs}
\usepackage{csvsimple}
\usepackage{geometry}
\usepackage{siunitx}
\usepackage{caption}
\usepackage{booktabs}
\usepackage{longtable}
\usepackage{float}
\usepackage{algpseudocode}
\usepackage{algorithm}
\usepackage{csquotes}
\usepackage{hyperref}



\captionsetup[table]{skip=10pt}

\sisetup{round-mode=places, round-precision=3, round-integer-to-decimal, scientific-notation = true, table-format = 1.3e2, table-number-alignment=center, group-separator={,}}


\title{Inżynieria uczenia maszynowego - projekt}
\author{Tomasz Owienko \and Anna Schäfer}
\date{29.11.2023}


% definicja problemu biznesowego
% zadania modelowania
% założenia
% kryteria sukcesu
% analiza danych z perspektywy realizacji zadań


\begin{document}
\maketitle

\section{Temat projektu}




Temat projektu przekazany przez Klienta:

\begin{displayquote}
	\textit{Może bylibyśmy w stanie wygenerować playlistę, która spodoba się kilku wybranym osobom jednocześnie? Coraz więcej osób używa Pozytywki podczas różnego rodzaju imprez i taka funkcjonalność byłaby hitem!}
\end{displayquote}


\section{Problem biznesowy}

Klientem jest właściciel portalu "Pozytywka", będącego serwisem muzycznym, pozwalającym użytkownikom na odtwarzanie utworów online.

Celem projektu jest realizacja funkcjonalności pozwalającej użytkownikom serwisu Pozytywka na generowanie playlist, z których utwory podobać się będą wybranej grupie użytkowników. Taka funkcjonalność mogłaby być wykorzystywana do automatycznego układania playlist na imprezy w taki sposób, aby ich zawartość trafiała w gust jak największej części odbiorców. Implementacja takiej funkcjonalności ma zwiększyć zadowolenie użytkowników z jakości playlist odtwarzanych na imprezach, tym samym zwiększając ich zadowolenie z użytkowania portalu.

\subsection*{Biznesowe kryterium sukcesu}
\begin{itemize}
\item W co najmniej $1/4$ uruchomień, playlista zostanie odtworzona przez minimum $35\%$ jej czasu trwania w ciągu jednej z najbliższych $20$ sesji jednego z użytkowników, który brał udział w tworzeniu playlisty.
\end{itemize}



\subsection{Założenia}
\begin{itemize}
	\item Playlisty generowane będą na podstawie profili oraz historii sesji nie więcej niż $10$ użytkowników jednocześnie,
	\item playlisty w większości przypadków użycia nie będą wykorzystywane wielokrotnie,
	\item dobór kolejności utworów na playliście nie jest przedmiotem zadania,
	\item dostęp do playlisty mają wszyscy użytkownicy, których profile i historia sesji były uwzględnione przy jej generowaniu,
	\item generowane playlisty składają się z 20 utworów każda.
\end{itemize}

\subsection{Pożądane cechy rozwiązania}
\begin{itemize}

\item Playlista może być wygenerowana w bardzo krótkim czasie,

\item funkcjonalność zachowuje się poprawnie dla nowo dodanych użytkowników oraz utworów,

\item w ocenianiu gustu muzycznego poszczególnych użytkowników większe znaczenie powinny mieć niedawno odtwarzane utwory.

\end{itemize}

\section{Zadanie modelowania}
	
Projekt zakłada zamodelowanie problemu jako zadanie generowania rekomendacji. Planowane jest zastosowanie podejścia \textit{collaborative filtering}, które (w kontekście zadania) opiera się na wyszukiwaniu użytkowników podobnych do rozpatrywanych i generowania rekomendacji w oparciu o ich historie sesji. Do realizacji podejścia \textit{collaborative filtering} zastosowana zostanie technika rozkładu macierzy interakcji między użytkownikami, a utworami. Przewidziane jest porównanie jakości modelu korzystającego z macierzy \textit{feedbacku niejawnego} (użytkownik $X$ odtworzył utwór $Y$), oraz \textit{feedbacku jawnego} (użytkownik $X$ wystawił utworowi $Y$ ocenę $Z$).

Podejście \textit{collaborative filtering} pozwala na generowanie rekomendacji dla pojedynczego użytkownika. Aby dostosować je do problemu, generowanie rekomendacji dla wielu użytkowników jednocześnie zamodelowane będzie jako:
\begin{enumerate}

\item wygenerowanie bardzo dużej liczby rekomendacji dla każdego z użytkowników wraz z oceną podobieństwa poszczególnych utworów do gustu muzycznego użytkownika (liczba rekomendacji proporcjonalna do liczby użytkowników),
\item znormalizowanie ocen podobieństwa do zakresu $<0, 1>$,
\item wyznacznie zbioru utworów, które pojawiły się w rekomendacjach wszystkich użytkowników,
\item wybór tych utworów, dla których iloczyn ich ocen dla wszystkich użytkowników jest największy.

\end{enumerate}

Istotnym problemem w rozważanym zadaniu jest tzw. \textit{cold-start}, czyli zachowanie modelu dla użytkowników bądź utworów, na których model nie był trenowany. Tradycyjne podejścia rozkładu macierzy interakcji takie jak \textit{Funk MF} czy \textit{SVD++} nie przewidują występowania takich sytuacji i spisują się słabo w scenariuszach \textit{cold-start}. W rozwiązaniu zostanie wykorzystany model \href{https://arxiv.org/pdf/1507.08439.pdf}{LightFM}, który rozwiązuje problem \textit{cold-start} przez zastosowanie metadanych (atrybutów) do opisywania zarówno użytkowników, jak i utworów. Przykładowo, nowy utwór dodany do systemu nie był brany pod uwagę przy trenowaniu modelu, ale posiada on atrybuty takie jak \texttt{instrumentalness}, \texttt{tempo}, \texttt{danceability} (zgodne z taksonomią Spotify), oraz wiele innych, co można wykorzystać do wyszukiwania podobieństw. 

\subsection*{Analityczne kryterium sukcesu}

Wyznaczanie rekomendacji dla pojedynczego użytkownika - pole pod krzywą ROC powyżej $0,6$ przy szacowaniu oceny dla utworów ze zbioru testowego (kryterium bezpośrednio związane z modelem)

Wyznaczanie rekomendacji dla wielu ($n$) użytkowników - co najmniej $50\%$ rekomendacji uwzględnionych w playliście wyznaczonych z pewnością $0.5^n$ przy w pełni losowym wyborze użytkowników.

Porównanie modelu docelowego z modelem bazowym (który to będzie polegał na wyborze losowych utworów z hisotrii poszczególnych użytkowników biorących udział w tworzeniu playlisty) - w modelu docelowym nastąpi lepsza klasteryzacja, to znaczy mniejsza średnia odległość utworu od środka klastra w przestrzeni embeddingów LightFMa.

\section{Analiza dostarczonych danych}


Z perspektywy rozwiązania problemu za pomocą podejścia \textit{collaborative filtering} kwestią kluczową jest pozyskanie dużej ilości informacji na temat użytkowników i historii sesji - zbyt mała ilość nie pozwoli na precyzyjne wyszukiwanie podobieństw pomiędzy użytkownikami. W trzeciej wersji otrzymanych danych uzyskano dane na temat $1100$ użytkowników oraz blisko $102.000$ sesji, co wstępnie uznano za ilość wystarczającą.


Model LightFM pozwala na wykorzystanie metadanych do opisu utworów i użytkowników, aby rozwiązać problem \textit{cold-start}. W przypadku użytkowników, wykorzystać można informację o zadeklarowanych przez nich preferencjach co do gatunków muzycznych. Należy mieć na uwadze, że sama deklaracja może nie być wystarczająca do znalezienia podobnych użytkowników - np w sytuacji, kiedy dana osoba posiada bardzo ubogą historię sesji, a ponadto zadeklaruje wyjątkowo nietypową kombinację preferowanych gatunków. W tym wypadku za preferencje muzyczne takiej osoby przyjęte zostaną utwory, które w ogólności charakteryzują się dużą popularnością, a ich waga przy wyznaczaniu końcowych rekomendacji zostanie ograniczona. 

Z perspektywy wytestowania podejścia opartego na \textit{jawnym feedbacku} należy rozważyć sposób zamodelowania takiej informacji - nie jest ona dana bezpośrednio w zbiorze danych. Można ją jednak oszacować wyznaczyć przez analizę historii sesji użytkownika - jednym z możliwych podejść może być wyznaczenie domniemanej oceny wystawionej przez użytkownika danemu utworowi na podstawie:

\begin{itemize}

\item częstotliwości odtwarzania danego utworu,
\item częstotliwości występowania zdarzenia \texttt{like},
\item częstotliwości występowania zdarzenia \texttt{skip}.

\end{itemize}

\subsection{Pierwsza iteracja zbierania danych}
Zostały odkryte liczne braki lub błędy w danych:
\begin{itemize}
\item \texttt{id=-1} oraz \texttt{genres=null} w pliku \texttt{artists.jsonl},
\item \texttt{id=null} w pliku \texttt{tracks.jsonl},
\item \texttt{id=null} w pliku \texttt{sessions.jsonl},
\item \texttt{user\_id=null}, \texttt{event\_type=null}, oraz \texttt{track\_id=null} dla \texttt{event\_type!=advertisement} w pliku\\ \texttt{sessions.jsonl},
\item sekwencje wierszy, gdzie \texttt{session\_id} oraz \texttt{timestamp} mają tę samą wartość w pliku \texttt{sessions.jsonl},
\item \texttt{favourite\_genres=null} w pliku \texttt{users.jsonl}.

\end{itemize}

Przy okazji prośby o nowe dane, uzgodniono znaczenie atrybutów i ich wartości w pliku \texttt{tracks.jsonl} oraz przyczynę obecności małej liczby wierszy z wartością \texttt{storage\_class=fast} w pliku \texttt{track\_storage.jsonl}.

\subsection{Druga iteracja zbierania danych}
\begin{itemize}
\item Otrzymano szczegółowe informacje dotyczące znaczenia atrybutów utworów z pliku \texttt{tracks.jsonl},
\item Klient wyeliminował braki oraz błędy wymienione w poprzedniej iteracji danych,
\item Został dostrzeżony fakt, że w pliku \texttt{tracks.jsonl} niektóre utwory występują kilkukrotnie, a owe wystąpienia różnią się jedynie wartościami \texttt{id} i \texttt{popularity} - zwrócono na to uwagę Klientowi.

\end{itemize}

Na tym etapie podjęto decyzję o skorzystaniu z podejścia collaborative filtering do rozwiązania problemu. Niezbędne było uzyskanie od Klienta większej ilości danych na temat użytkowników oraz ich historii sesji.

\subsection{Trzecia iteracja zbierania danych}
\begin{itemize}
\item Wyjaśniono wielokrotne wystąpienia tego samego utworu w pliku \texttt{tracks.jsonl} - kilkukrotne wystąpienie tego samego utworu pod innym \texttt{id} i \texttt{popularity} nie jest błędem, a wynika z kilkukrotnego pojawienia się utworu na rynku w różnych wydaniach,
\item Uzyskano większą ilość danych na temat użytkowników oraz ich historii sesji - wstępnie uznano ją za wystarczającą.
\end{itemize}

\end{document}

